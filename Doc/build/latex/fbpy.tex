% Generated by Sphinx.
\def\sphinxdocclass{report}
\documentclass[letterpaper,10pt,english]{sphinxmanual}
\usepackage[utf8]{inputenc}
\DeclareUnicodeCharacter{00A0}{\nobreakspace}
\usepackage[T1]{fontenc}
\usepackage{babel}
\usepackage{times}
\usepackage[Bjarne]{fncychap}
\usepackage{longtable}
\usepackage{sphinx}
\usepackage{multirow}


\title{fbpy Documentation}
\date{June 27, 2014}
\release{0.1}
\author{Marcell Marosvolgyi}
\newcommand{\sphinxlogo}{}
\renewcommand{\releasename}{Release}
\makeindex

\makeatletter
\def\PYG@reset{\let\PYG@it=\relax \let\PYG@bf=\relax%
    \let\PYG@ul=\relax \let\PYG@tc=\relax%
    \let\PYG@bc=\relax \let\PYG@ff=\relax}
\def\PYG@tok#1{\csname PYG@tok@#1\endcsname}
\def\PYG@toks#1+{\ifx\relax#1\empty\else%
    \PYG@tok{#1}\expandafter\PYG@toks\fi}
\def\PYG@do#1{\PYG@bc{\PYG@tc{\PYG@ul{%
    \PYG@it{\PYG@bf{\PYG@ff{#1}}}}}}}
\def\PYG#1#2{\PYG@reset\PYG@toks#1+\relax+\PYG@do{#2}}

\def\PYG@tok@gd{\def\PYG@tc##1{\textcolor[rgb]{0.63,0.00,0.00}{##1}}}
\def\PYG@tok@gu{\let\PYG@bf=\textbf\def\PYG@tc##1{\textcolor[rgb]{0.50,0.00,0.50}{##1}}}
\def\PYG@tok@gt{\def\PYG@tc##1{\textcolor[rgb]{0.00,0.25,0.82}{##1}}}
\def\PYG@tok@gs{\let\PYG@bf=\textbf}
\def\PYG@tok@gr{\def\PYG@tc##1{\textcolor[rgb]{1.00,0.00,0.00}{##1}}}
\def\PYG@tok@cm{\let\PYG@it=\textit\def\PYG@tc##1{\textcolor[rgb]{0.25,0.50,0.56}{##1}}}
\def\PYG@tok@vg{\def\PYG@tc##1{\textcolor[rgb]{0.73,0.38,0.84}{##1}}}
\def\PYG@tok@m{\def\PYG@tc##1{\textcolor[rgb]{0.13,0.50,0.31}{##1}}}
\def\PYG@tok@mh{\def\PYG@tc##1{\textcolor[rgb]{0.13,0.50,0.31}{##1}}}
\def\PYG@tok@cs{\def\PYG@tc##1{\textcolor[rgb]{0.25,0.50,0.56}{##1}}\def\PYG@bc##1{\colorbox[rgb]{1.00,0.94,0.94}{##1}}}
\def\PYG@tok@ge{\let\PYG@it=\textit}
\def\PYG@tok@vc{\def\PYG@tc##1{\textcolor[rgb]{0.73,0.38,0.84}{##1}}}
\def\PYG@tok@il{\def\PYG@tc##1{\textcolor[rgb]{0.13,0.50,0.31}{##1}}}
\def\PYG@tok@go{\def\PYG@tc##1{\textcolor[rgb]{0.19,0.19,0.19}{##1}}}
\def\PYG@tok@cp{\def\PYG@tc##1{\textcolor[rgb]{0.00,0.44,0.13}{##1}}}
\def\PYG@tok@gi{\def\PYG@tc##1{\textcolor[rgb]{0.00,0.63,0.00}{##1}}}
\def\PYG@tok@gh{\let\PYG@bf=\textbf\def\PYG@tc##1{\textcolor[rgb]{0.00,0.00,0.50}{##1}}}
\def\PYG@tok@ni{\let\PYG@bf=\textbf\def\PYG@tc##1{\textcolor[rgb]{0.84,0.33,0.22}{##1}}}
\def\PYG@tok@nl{\let\PYG@bf=\textbf\def\PYG@tc##1{\textcolor[rgb]{0.00,0.13,0.44}{##1}}}
\def\PYG@tok@nn{\let\PYG@bf=\textbf\def\PYG@tc##1{\textcolor[rgb]{0.05,0.52,0.71}{##1}}}
\def\PYG@tok@no{\def\PYG@tc##1{\textcolor[rgb]{0.38,0.68,0.84}{##1}}}
\def\PYG@tok@na{\def\PYG@tc##1{\textcolor[rgb]{0.25,0.44,0.63}{##1}}}
\def\PYG@tok@nb{\def\PYG@tc##1{\textcolor[rgb]{0.00,0.44,0.13}{##1}}}
\def\PYG@tok@nc{\let\PYG@bf=\textbf\def\PYG@tc##1{\textcolor[rgb]{0.05,0.52,0.71}{##1}}}
\def\PYG@tok@nd{\let\PYG@bf=\textbf\def\PYG@tc##1{\textcolor[rgb]{0.33,0.33,0.33}{##1}}}
\def\PYG@tok@ne{\def\PYG@tc##1{\textcolor[rgb]{0.00,0.44,0.13}{##1}}}
\def\PYG@tok@nf{\def\PYG@tc##1{\textcolor[rgb]{0.02,0.16,0.49}{##1}}}
\def\PYG@tok@si{\let\PYG@it=\textit\def\PYG@tc##1{\textcolor[rgb]{0.44,0.63,0.82}{##1}}}
\def\PYG@tok@s2{\def\PYG@tc##1{\textcolor[rgb]{0.25,0.44,0.63}{##1}}}
\def\PYG@tok@vi{\def\PYG@tc##1{\textcolor[rgb]{0.73,0.38,0.84}{##1}}}
\def\PYG@tok@nt{\let\PYG@bf=\textbf\def\PYG@tc##1{\textcolor[rgb]{0.02,0.16,0.45}{##1}}}
\def\PYG@tok@nv{\def\PYG@tc##1{\textcolor[rgb]{0.73,0.38,0.84}{##1}}}
\def\PYG@tok@s1{\def\PYG@tc##1{\textcolor[rgb]{0.25,0.44,0.63}{##1}}}
\def\PYG@tok@gp{\let\PYG@bf=\textbf\def\PYG@tc##1{\textcolor[rgb]{0.78,0.36,0.04}{##1}}}
\def\PYG@tok@sh{\def\PYG@tc##1{\textcolor[rgb]{0.25,0.44,0.63}{##1}}}
\def\PYG@tok@ow{\let\PYG@bf=\textbf\def\PYG@tc##1{\textcolor[rgb]{0.00,0.44,0.13}{##1}}}
\def\PYG@tok@sx{\def\PYG@tc##1{\textcolor[rgb]{0.78,0.36,0.04}{##1}}}
\def\PYG@tok@bp{\def\PYG@tc##1{\textcolor[rgb]{0.00,0.44,0.13}{##1}}}
\def\PYG@tok@c1{\let\PYG@it=\textit\def\PYG@tc##1{\textcolor[rgb]{0.25,0.50,0.56}{##1}}}
\def\PYG@tok@kc{\let\PYG@bf=\textbf\def\PYG@tc##1{\textcolor[rgb]{0.00,0.44,0.13}{##1}}}
\def\PYG@tok@c{\let\PYG@it=\textit\def\PYG@tc##1{\textcolor[rgb]{0.25,0.50,0.56}{##1}}}
\def\PYG@tok@mf{\def\PYG@tc##1{\textcolor[rgb]{0.13,0.50,0.31}{##1}}}
\def\PYG@tok@err{\def\PYG@bc##1{\fcolorbox[rgb]{1.00,0.00,0.00}{1,1,1}{##1}}}
\def\PYG@tok@kd{\let\PYG@bf=\textbf\def\PYG@tc##1{\textcolor[rgb]{0.00,0.44,0.13}{##1}}}
\def\PYG@tok@ss{\def\PYG@tc##1{\textcolor[rgb]{0.32,0.47,0.09}{##1}}}
\def\PYG@tok@sr{\def\PYG@tc##1{\textcolor[rgb]{0.14,0.33,0.53}{##1}}}
\def\PYG@tok@mo{\def\PYG@tc##1{\textcolor[rgb]{0.13,0.50,0.31}{##1}}}
\def\PYG@tok@mi{\def\PYG@tc##1{\textcolor[rgb]{0.13,0.50,0.31}{##1}}}
\def\PYG@tok@kn{\let\PYG@bf=\textbf\def\PYG@tc##1{\textcolor[rgb]{0.00,0.44,0.13}{##1}}}
\def\PYG@tok@o{\def\PYG@tc##1{\textcolor[rgb]{0.40,0.40,0.40}{##1}}}
\def\PYG@tok@kr{\let\PYG@bf=\textbf\def\PYG@tc##1{\textcolor[rgb]{0.00,0.44,0.13}{##1}}}
\def\PYG@tok@s{\def\PYG@tc##1{\textcolor[rgb]{0.25,0.44,0.63}{##1}}}
\def\PYG@tok@kp{\def\PYG@tc##1{\textcolor[rgb]{0.00,0.44,0.13}{##1}}}
\def\PYG@tok@w{\def\PYG@tc##1{\textcolor[rgb]{0.73,0.73,0.73}{##1}}}
\def\PYG@tok@kt{\def\PYG@tc##1{\textcolor[rgb]{0.56,0.13,0.00}{##1}}}
\def\PYG@tok@sc{\def\PYG@tc##1{\textcolor[rgb]{0.25,0.44,0.63}{##1}}}
\def\PYG@tok@sb{\def\PYG@tc##1{\textcolor[rgb]{0.25,0.44,0.63}{##1}}}
\def\PYG@tok@k{\let\PYG@bf=\textbf\def\PYG@tc##1{\textcolor[rgb]{0.00,0.44,0.13}{##1}}}
\def\PYG@tok@se{\let\PYG@bf=\textbf\def\PYG@tc##1{\textcolor[rgb]{0.25,0.44,0.63}{##1}}}
\def\PYG@tok@sd{\let\PYG@it=\textit\def\PYG@tc##1{\textcolor[rgb]{0.25,0.44,0.63}{##1}}}

\def\PYGZbs{\char`\\}
\def\PYGZus{\char`\_}
\def\PYGZob{\char`\{}
\def\PYGZcb{\char`\}}
\def\PYGZca{\char`\^}
\def\PYGZsh{\char`\#}
\def\PYGZpc{\char`\%}
\def\PYGZdl{\char`\$}
\def\PYGZti{\char`\~}
% for compatibility with earlier versions
\def\PYGZat{@}
\def\PYGZlb{[}
\def\PYGZrb{]}
\makeatother

\begin{document}

\maketitle
\tableofcontents
\phantomsection\label{index::doc}


Contents:


\chapter{General description}
\label{index:general-description}\label{index:welcome-to-fbpy-s-documentation}
The \emph{fbpy} module is an API for drawing in the framebuffer on Linux machines.
It was conceived as part of an audio player project based on the
raspberry pi computer and wolfson pi audio interface. I needed a low-level
graphics library for visualizing audio data (scope, phase,...). I also
wanted to gain some programming skills, like writing c libs for python and
some kernel stuff. So this module is by no means an attempt to make a
better graphics lib with fancy hardware acceleration or anythin or
making something original. I think it is use able though and by examining
the source, it might serve as a form of documentation if you want to
make something like this yourself. That is why I publish it. Oh, and of course
because I support open source hardware \emph{and} software, the `firmware' of
my audio player should be available as source :)


\section{Website}
\label{index:website}
\href{http://transistorlove.wordpress.com}{http://transistorlove.wordpress.com}


\chapter{Module documentation}
\label{index:module-fb}\label{index:module-documentation}\index{fb (module)}\index{Colors (class in fb)}

\begin{fulllineitems}
\phantomsection\label{index:fb.Colors}\pysigline{\strong{class }\code{fb.}\bfcode{Colors}}
Some prefab colors, to make life easier.

Food for Pixelstyle. e.g.:

\end{fulllineitems}

\index{Surface (class in fb)}

\begin{fulllineitems}
\phantomsection\label{index:fb.Surface}\pysiglinewithargsret{\strong{class }\code{fb.}\bfcode{Surface}}{\emph{*args}}{}
This is the main class, it generates a drawing surface.

On first invokation, it will generate a surface which
encompasses the entire screen automaticaly \emph{and} it
will open the framebuffer device.
The \emph{classmethod} close will close it. 
Subsequent instances will need arguments defining size and
position.
\index{addpoly() (fb.Surface static method)}

\begin{fulllineitems}
\phantomsection\label{index:fb.Surface.addpoly}\pysiglinewithargsret{\strong{static }\bfcode{addpoly}}{\emph{*args}, \emph{**kwargs}}{}
just a test for the moment

addpoly(\textless{}x array\textgreater{},\textless{}y array\textgreater{})

\end{fulllineitems}

\index{arc() (fb.Surface static method)}

\begin{fulllineitems}
\phantomsection\label{index:fb.Surface.arc}\pysiglinewithargsret{\strong{static }\bfcode{arc}}{\emph{\textless{}tuple\textgreater{}}, \emph{\textless{}radius 1\textgreater{}}, \emph{\textless{}radius 2\textgreater{}}, \emph{\textless{}start seg\textgreater{}}, \emph{\textless{}end seg\textgreater{}}, \emph{\textless{}no seg\textgreater{}}}{}
couple of examples here:

\begin{Verbatim}[commandchars=\\\{\}]
\PYG{g+gp}{\textgreater{}\textgreater{}\textgreater{} }\PYG{k+kn}{import} \PYG{n+nn}{fbpy.fb} \PYG{k+kn}{as} \PYG{n+nn}{fb}

\PYG{g+gp}{\textgreater{}\textgreater{}\textgreater{} }\PYG{n}{main} \PYG{o}{=} \PYG{n}{fb}\PYG{o}{.}\PYG{n}{Surface}\PYG{p}{(}\PYG{p}{)}

\PYG{g+gp}{\textgreater{}\textgreater{}\textgreater{} }\PYG{n}{sub} \PYG{o}{=} \PYG{n}{fb}\PYG{o}{.}\PYG{n}{Surface}\PYG{p}{(}\PYG{p}{(}\PYG{l+m+mi}{0}\PYG{p}{,}\PYG{l+m+mi}{0}\PYG{p}{)}\PYG{p}{,} \PYG{p}{(}\PYG{l+m+mi}{200}\PYG{p}{,}\PYG{l+m+mi}{200}\PYG{p}{)}\PYG{p}{)}

\PYG{g+gp}{\textgreater{}\textgreater{}\textgreater{} }\PYG{n}{sub}\PYG{o}{.}\PYG{n}{clear}\PYG{p}{(}\PYG{p}{)}
\PYG{g+go}{0}
\PYG{g+gp}{\textgreater{}\textgreater{}\textgreater{} }\PYG{n}{sub}\PYG{o}{.}\PYG{n}{pixelstyle} \PYG{o}{=} \PYG{n}{fb}\PYG{o}{.}\PYG{n}{Pixelstyles}\PYG{o}{.}\PYG{n}{faint}

\PYG{g+gp}{\textgreater{}\textgreater{}\textgreater{} }\PYG{n}{sub}\PYG{o}{.}\PYG{n}{arc}\PYG{p}{(}\PYG{p}{(}\PYG{l+m+mi}{100}\PYG{p}{,}\PYG{l+m+mi}{100}\PYG{p}{)}\PYG{p}{,} \PYG{l+m+mi}{60}\PYG{p}{,} \PYG{l+m+mi}{90}\PYG{p}{,} \PYG{l+m+mi}{0}\PYG{p}{,} \PYG{l+m+mi}{50}\PYG{p}{,} \PYG{l+m+mi}{100}\PYG{p}{)}
\PYG{g+go}{0}
\PYG{g+gp}{\textgreater{}\textgreater{}\textgreater{} }\PYG{n}{sub}\PYG{o}{.}\PYG{n}{pixelstyle} \PYG{o}{=} \PYG{n}{fb}\PYG{o}{.}\PYG{n}{Pixelstyles}\PYG{o}{.}\PYG{n}{sharp}

\PYG{g+gp}{\textgreater{}\textgreater{}\textgreater{} }\PYG{n}{sub}\PYG{o}{.}\PYG{n}{arc}\PYG{p}{(}\PYG{p}{(}\PYG{l+m+mi}{100}\PYG{p}{,}\PYG{l+m+mi}{100}\PYG{p}{)}\PYG{p}{,} \PYG{l+m+mi}{40}\PYG{p}{,} \PYG{l+m+mi}{40}\PYG{p}{,} \PYG{l+m+mi}{30}\PYG{p}{,} \PYG{l+m+mi}{90}\PYG{p}{,} \PYG{l+m+mi}{100}\PYG{p}{)}
\PYG{g+go}{0}
\PYG{g+gp}{\textgreater{}\textgreater{}\textgreater{} }\PYG{n}{sub}\PYG{o}{.}\PYG{n}{grabsilent}\PYG{p}{(}\PYG{l+s}{"}\PYG{l+s}{./source/images/arc.png}\PYG{l+s}{"}\PYG{p}{)}
\PYG{g+go}{0}
\end{Verbatim}

\includegraphics{arc.png}

\end{fulllineitems}

\index{blit() (fb.Surface method)}

\begin{fulllineitems}
\phantomsection\label{index:fb.Surface.blit}\pysiglinewithargsret{\bfcode{blit}}{\emph{\textless{}filename\textgreater{}}}{}
will put the PNG \textless{}filename\textgreater{} in the current surface

\begin{Verbatim}[commandchars=\\\{\}]
\PYG{g+gp}{\textgreater{}\textgreater{}\textgreater{} }\PYG{k+kn}{import} \PYG{n+nn}{fbpy.fb} \PYG{k+kn}{as} \PYG{n+nn}{fb}

\PYG{g+gp}{\textgreater{}\textgreater{}\textgreater{} }\PYG{n}{main} \PYG{o}{=} \PYG{n}{fb}\PYG{o}{.}\PYG{n}{Surface}\PYG{p}{(}\PYG{p}{)}

\PYG{g+gp}{\textgreater{}\textgreater{}\textgreater{} }\PYG{n}{sub} \PYG{o}{=} \PYG{n}{fb}\PYG{o}{.}\PYG{n}{Surface}\PYG{p}{(}\PYG{p}{(}\PYG{l+m+mi}{100}\PYG{p}{,}\PYG{l+m+mi}{100}\PYG{p}{)}\PYG{p}{,}\PYG{p}{(}\PYG{l+m+mi}{600}\PYG{p}{,}\PYG{l+m+mi}{600}\PYG{p}{)}\PYG{p}{)}

\PYG{g+gp}{\textgreater{}\textgreater{}\textgreater{} }\PYG{n}{sub}\PYG{o}{.}\PYG{n}{blit}\PYG{p}{(}\PYG{l+s}{"}\PYG{l+s}{../examples/cylon.png}\PYG{l+s}{"}\PYG{p}{)}
\PYG{g+go}{0}
\PYG{g+gp}{\textgreater{}\textgreater{}\textgreater{} }\PYG{n}{sub}\PYG{o}{.}\PYG{n}{grabsilent}\PYG{p}{(}\PYG{l+s}{"}\PYG{l+s}{./source/images/gottherobot.png}\PYG{l+s}{"}\PYG{p}{)}
\PYG{g+go}{0}
\end{Verbatim}

\includegraphics{gottherobot.png}

\end{fulllineitems}

\index{circle() (fb.Surface static method)}

\begin{fulllineitems}
\phantomsection\label{index:fb.Surface.circle}\pysiglinewithargsret{\strong{static }\bfcode{circle}}{\emph{\textless{}tuple\textgreater{}}, \emph{\textless{}radius\textgreater{}}, \emph{\textless{}segments\textgreater{}}}{}
Will draw a ...

\begin{Verbatim}[commandchars=\\\{\}]
\PYG{g+gp}{\textgreater{}\textgreater{}\textgreater{} }\PYG{k+kn}{import} \PYG{n+nn}{fbpy.fb} \PYG{k+kn}{as} \PYG{n+nn}{fb}

\PYG{g+gp}{\textgreater{}\textgreater{}\textgreater{} }\PYG{n}{main} \PYG{o}{=} \PYG{n}{fb}\PYG{o}{.}\PYG{n}{Surface}\PYG{p}{(}\PYG{p}{)}

\PYG{g+gp}{\textgreater{}\textgreater{}\textgreater{} }\PYG{n}{sub} \PYG{o}{=} \PYG{n}{fb}\PYG{o}{.}\PYG{n}{Surface}\PYG{p}{(}\PYG{p}{(}\PYG{l+m+mi}{0}\PYG{p}{,}\PYG{l+m+mi}{0}\PYG{p}{)}\PYG{p}{,} \PYG{p}{(}\PYG{l+m+mi}{200}\PYG{p}{,}\PYG{l+m+mi}{200}\PYG{p}{)}\PYG{p}{)}

\PYG{g+gp}{\textgreater{}\textgreater{}\textgreater{} }\PYG{n}{sub}\PYG{o}{.}\PYG{n}{clear}\PYG{p}{(}\PYG{p}{)}
\PYG{g+go}{0}
\PYG{g+gp}{\textgreater{}\textgreater{}\textgreater{} }\PYG{n}{sub}\PYG{o}{.}\PYG{n}{circle}\PYG{p}{(}\PYG{p}{(}\PYG{l+m+mi}{100}\PYG{p}{,}\PYG{l+m+mi}{100}\PYG{p}{)}\PYG{p}{,}\PYG{l+m+mf}{0.5}\PYG{p}{,} \PYG{l+m+mi}{100}\PYG{p}{)}
\PYG{g+go}{0}
\PYG{g+gp}{\textgreater{}\textgreater{}\textgreater{} }\PYG{n}{sub}\PYG{o}{.}\PYG{n}{grabsilent}\PYG{p}{(}\PYG{l+s}{"}\PYG{l+s}{./source/images/circle.png}\PYG{l+s}{"}\PYG{p}{)}
\PYG{g+go}{0}
\end{Verbatim}

\includegraphics{circle.png}

\end{fulllineitems}

\index{clear() (fb.Surface method)}

\begin{fulllineitems}
\phantomsection\label{index:fb.Surface.clear}\pysiglinewithargsret{\bfcode{clear}}{}{}
will make blackscreen

\end{fulllineitems}

\index{drawpolys() (fb.Surface method)}

\begin{fulllineitems}
\phantomsection\label{index:fb.Surface.drawpolys}\pysiglinewithargsret{\bfcode{drawpolys}}{}{}
Draw a bunch of polygons

\begin{Verbatim}[commandchars=\\\{\}]
\PYG{g+gp}{\textgreater{}\textgreater{}\textgreater{} }\PYG{k+kn}{import} \PYG{n+nn}{fbpy.fb} \PYG{k+kn}{as} \PYG{n+nn}{fb}

\PYG{g+gp}{\textgreater{}\textgreater{}\textgreater{} }\PYG{k+kn}{import} \PYG{n+nn}{numpy} \PYG{k+kn}{as} \PYG{n+nn}{np}

\PYG{g+gp}{\textgreater{}\textgreater{}\textgreater{} }\PYG{n}{main}  \PYG{o}{=} \PYG{n}{fb}\PYG{o}{.}\PYG{n}{Surface}\PYG{p}{(}\PYG{p}{)}

\PYG{g+gp}{\textgreater{}\textgreater{}\textgreater{} }\PYG{n}{main}\PYG{o}{.}\PYG{n}{clear}\PYG{p}{(}\PYG{p}{)}

\PYG{g+gp}{\textgreater{}\textgreater{}\textgreater{} }\PYG{n}{sub} \PYG{o}{=} \PYG{n}{fb}\PYG{o}{.}\PYG{n}{Surface}\PYG{p}{(}\PYG{p}{(}\PYG{l+m+mi}{100}\PYG{p}{,}\PYG{l+m+mi}{100}\PYG{p}{)}\PYG{p}{,}\PYG{p}{(}\PYG{l+m+mi}{200}\PYG{p}{,}\PYG{l+m+mi}{200}\PYG{p}{)}\PYG{p}{)}

\PYG{g+gp}{\textgreater{}\textgreater{}\textgreater{} }\PYG{n}{sub}\PYG{o}{.}\PYG{n}{clear}\PYG{p}{(}\PYG{p}{)}
\PYG{g+go}{0}
\PYG{g+gp}{\textgreater{}\textgreater{}\textgreater{} }\PYG{n}{x1} \PYG{o}{=} \PYG{n}{np}\PYG{o}{.}\PYG{n}{arange}\PYG{p}{(}\PYG{l+m+mi}{0}\PYG{p}{,}\PYG{l+m+mi}{1}\PYG{p}{,}\PYG{l+m+mf}{0.1}\PYG{p}{)}

\PYG{g+gp}{\textgreater{}\textgreater{}\textgreater{} }\PYG{n}{y1} \PYG{o}{=} \PYG{l+m+mf}{0.5}\PYG{o}{*}\PYG{n}{np}\PYG{o}{.}\PYG{n}{sin}\PYG{p}{(}\PYG{n}{x1}\PYG{o}{*}\PYG{l+m+mi}{2}\PYG{o}{*}\PYG{n}{np}\PYG{o}{.}\PYG{n}{pi}\PYG{p}{)}\PYG{o}{+}\PYG{l+m+mf}{0.5}

\PYG{g+gp}{\textgreater{}\textgreater{}\textgreater{} }\PYG{n}{x2} \PYG{o}{=} \PYG{n}{np}\PYG{o}{.}\PYG{n}{arange}\PYG{p}{(}\PYG{l+m+mi}{0}\PYG{p}{,}\PYG{l+m+mi}{1}\PYG{p}{,}\PYG{l+m+mf}{0.1}\PYG{p}{)}

\PYG{g+gp}{\textgreater{}\textgreater{}\textgreater{} }\PYG{n}{y2} \PYG{o}{=} \PYG{l+m+mf}{0.5}\PYG{o}{*}\PYG{n}{np}\PYG{o}{.}\PYG{n}{cos}\PYG{p}{(}\PYG{n}{x2}\PYG{o}{*}\PYG{l+m+mi}{2}\PYG{o}{*}\PYG{n}{np}\PYG{o}{.}\PYG{n}{pi}\PYG{p}{)}\PYG{o}{+}\PYG{l+m+mf}{0.5}

\PYG{g+gp}{\textgreater{}\textgreater{}\textgreater{} }\PYG{n}{sub}\PYG{o}{.}\PYG{n}{addpoly}\PYG{p}{(}\PYG{n}{x1}\PYG{p}{,}\PYG{n}{y1}\PYG{p}{)}
\PYG{g+go}{0}
\PYG{g+gp}{\textgreater{}\textgreater{}\textgreater{} }\PYG{n}{sub}\PYG{o}{.}\PYG{n}{addpoly}\PYG{p}{(}\PYG{n}{x2}\PYG{p}{,}\PYG{n}{y2}\PYG{p}{)}
\PYG{g+go}{0}
\PYG{g+gp}{\textgreater{}\textgreater{}\textgreater{} }\PYG{n}{sub}\PYG{o}{.}\PYG{n}{drawpolys}\PYG{p}{(}\PYG{p}{)}
\PYG{g+go}{0}
\PYG{g+gp}{\textgreater{}\textgreater{}\textgreater{} }\PYG{n}{sub}\PYG{o}{.}\PYG{n}{trafo}\PYG{o}{.}\PYG{n}{rotate}\PYG{p}{(}\PYG{l+m+mf}{0.1}\PYG{p}{)}
\PYG{g+go}{0}
\PYG{g+gp}{\textgreater{}\textgreater{}\textgreater{} }\PYG{n}{sub}\PYG{o}{.}\PYG{n}{drawpolys}\PYG{p}{(}\PYG{p}{)}
\PYG{g+go}{0}
\PYG{g+gp}{\textgreater{}\textgreater{}\textgreater{} }\PYG{n}{sub}\PYG{o}{.}\PYG{n}{grabsilent}\PYG{p}{(}\PYG{l+s}{"}\PYG{l+s}{./source/images/polys.png}\PYG{l+s}{"}\PYG{p}{)}
\PYG{g+go}{0}
\end{Verbatim}

\includegraphics{polys.png}

\end{fulllineitems}

\index{get\_raw() (fb.Surface method)}

\begin{fulllineitems}
\phantomsection\label{index:fb.Surface.get_raw}\pysiglinewithargsret{\bfcode{get\_raw}}{}{}
returns an raw bitmap array of the current window, use
set\_raw to put the bitmap back.

\begin{Verbatim}[commandchars=\\\{\}]
\PYG{n}{sprite} \PYG{o}{=} \PYG{n}{main}\PYG{o}{.}\PYG{n}{get\PYGZus{}raw}\PYG{p}{(}\PYG{p}{)}
\PYG{n}{main}\PYG{o}{.}\PYG{n}{set\PYGZus{}raw}\PYG{p}{(}\PYG{n}{sprite}\PYG{p}{)}
\end{Verbatim}

\end{fulllineitems}

\index{grab() (fb.Surface method)}

\begin{fulllineitems}
\phantomsection\label{index:fb.Surface.grab}\pysiglinewithargsret{\bfcode{grab}}{\emph{\textless{}filename\textgreater{}}}{}
grabs current frame into file \textless{}filename\textgreater{}.png

\end{fulllineitems}

\index{grabsequence() (fb.Surface method)}

\begin{fulllineitems}
\phantomsection\label{index:fb.Surface.grabsequence}\pysiglinewithargsret{\bfcode{grabsequence}}{\emph{\textless{}filename\textgreater{}}}{}
grabs current frame into file with filename \textless{}filename\#\textgreater{}

where \# is an automatich counter. the output will be e.g.:
screenshot0001.png, screenshot0002.png, ...

you can use e.g.

\begin{Verbatim}[commandchars=\\\{\}]
\PYG{g+gp}{nerd@wonka: \PYGZti{}/tmp\PYGZdl{}} avconv -i \textless{}filename\textgreater{}\PYGZpc{}04d.png -c:v huffyuv \textless{}yourmoviename\textgreater{}.avi
\end{Verbatim}

to convert the sequence to a movie.
You can also use ofcourse somehtin like

\begin{Verbatim}[commandchars=\\\{\}]
\PYG{g+gp}{nerd@wonka: \PYGZti{}/tmp\PYGZdl{}} avconv -f fbdev -r 10 -i /dev/fb0 -c:v huffyuv /dev/shm/movi.avi 2\textgreater{} /dev/null
\end{Verbatim}

\end{fulllineitems}

\index{grabsilent() (fb.Surface method)}

\begin{fulllineitems}
\phantomsection\label{index:fb.Surface.grabsilent}\pysiglinewithargsret{\bfcode{grabsilent}}{\emph{\textless{}filename\textgreater{}}}{}
grabs current buffer into file \textless{}filename\textgreater{}.png

so, if you dont use update, you'll never actually 
\emph{see} the drawing. Handy for doctest stuff 
of other apps where you \emph{only} wanna make 
pics..

\end{fulllineitems}

\index{graticule() (fb.Surface static method)}

\begin{fulllineitems}
\phantomsection\label{index:fb.Surface.graticule}\pysiglinewithargsret{\strong{static }\bfcode{graticule}}{\emph{\textless{}tuple\textgreater{}}, \emph{\textless{}tuple\textgreater{}}, \emph{\textless{}fb.color\textgreater{}}, \emph{\textless{}fb.color\textgreater{}}}{}
draws scope-like graticule @ first tuple of size second tuple
(width/height). color = subs, color2 main

returns 0

\begin{Verbatim}[commandchars=\\\{\}]
\PYG{g+gp}{\textgreater{}\textgreater{}\textgreater{} }\PYG{k+kn}{import} \PYG{n+nn}{fbpy.fb} \PYG{k+kn}{as} \PYG{n+nn}{fb}

\PYG{g+gp}{\textgreater{}\textgreater{}\textgreater{} }\PYG{n}{main}  \PYG{o}{=} \PYG{n}{fb}\PYG{o}{.}\PYG{n}{Surface}\PYG{p}{(}\PYG{p}{)}

\PYG{g+gp}{\textgreater{}\textgreater{}\textgreater{} }\PYG{n}{sub2} \PYG{o}{=} \PYG{n}{fb}\PYG{o}{.}\PYG{n}{Surface}\PYG{p}{(}\PYG{p}{(}\PYG{l+m+mi}{0}\PYG{p}{,}\PYG{l+m+mi}{0}\PYG{p}{)}\PYG{p}{,}\PYG{p}{(}\PYG{l+m+mi}{200}\PYG{p}{,}\PYG{l+m+mi}{200}\PYG{p}{)}\PYG{p}{)}

\PYG{g+gp}{\textgreater{}\textgreater{}\textgreater{} }\PYG{n}{sub2}\PYG{o}{.}\PYG{n}{clear}\PYG{p}{(}\PYG{p}{)} \PYG{o}{==} \PYG{l+m+mi}{0}
\PYG{g+go}{True}
\PYG{g+gp}{\textgreater{}\textgreater{}\textgreater{} }\PYG{n}{sub2}\PYG{o}{.}\PYG{n}{pixelstyle}\PYG{o}{.}\PYG{n}{color} \PYG{o}{=} \PYG{n}{fb}\PYG{o}{.}\PYG{n}{Color}\PYG{p}{(}\PYG{l+m+mi}{200}\PYG{p}{,}\PYG{l+m+mi}{200}\PYG{p}{,}\PYG{l+m+mi}{200}\PYG{p}{,}\PYG{l+m+mo}{00}\PYG{p}{)} 
\PYG{g+go}{  }
\PYG{g+gp}{\textgreater{}\textgreater{}\textgreater{} }\PYG{n}{sub2}\PYG{o}{.}\PYG{n}{fillrect}\PYG{p}{(}\PYG{p}{(}\PYG{l+m+mi}{0}\PYG{p}{,}\PYG{l+m+mi}{0}\PYG{p}{)}\PYG{p}{,}\PYG{p}{(}\PYG{l+m+mi}{200}\PYG{p}{,}\PYG{l+m+mi}{200}\PYG{p}{)}\PYG{p}{)} \PYG{o}{==} \PYG{l+m+mi}{0}
\PYG{g+go}{True}
\PYG{g+gp}{\textgreater{}\textgreater{}\textgreater{} }\PYG{n}{sub2}\PYG{o}{.}\PYG{n}{pixelstyle}\PYG{o}{.}\PYG{n}{color} \PYG{o}{=} \PYG{n}{fb}\PYG{o}{.}\PYG{n}{Colors}\PYG{o}{.}\PYG{n}{white}

\PYG{g+gp}{\textgreater{}\textgreater{}\textgreater{} }\PYG{n}{sub2}\PYG{o}{.}\PYG{n}{graticule}\PYG{p}{(}\PYG{p}{(}\PYG{l+m+mf}{0.0}\PYG{p}{,}\PYG{l+m+mf}{0.0}\PYG{p}{)}\PYG{p}{,}\PYG{p}{(}\PYG{l+m+mf}{1.0}\PYG{p}{,}\PYG{l+m+mf}{1.0}\PYG{p}{)}\PYG{p}{)} \PYG{o}{==} \PYG{l+m+mi}{0}
\PYG{g+go}{True}
\PYG{g+gp}{\textgreater{}\textgreater{}\textgreater{} }\PYG{n}{sub2}\PYG{o}{.}\PYG{n}{grabsilent}\PYG{p}{(}\PYG{l+s}{"}\PYG{l+s}{./source/images/graticule.png}\PYG{l+s}{"}\PYG{p}{)} \PYG{o}{==} \PYG{l+m+mi}{0}
\PYG{g+go}{True}
\end{Verbatim}

\includegraphics{graticule.png}

\end{fulllineitems}

\index{informdriver() (fb.Surface method)}

\begin{fulllineitems}
\phantomsection\label{index:fb.Surface.informdriver}\pysiglinewithargsret{\bfcode{informdriver}}{}{}
pass relevant class info to
fbutils driver,
this is how one `instance' of the
driver can serve multiple Surface
instances

\begin{Verbatim}[commandchars=\\\{\}]
\PYG{g+gp}{\textgreater{}\textgreater{}\textgreater{} }\PYG{k+kn}{import} \PYG{n+nn}{fbpy.fb} \PYG{k+kn}{as} \PYG{n+nn}{fb}

\PYG{g+gp}{\textgreater{}\textgreater{}\textgreater{} }\PYG{n}{main} \PYG{o}{=} \PYG{n}{fb}\PYG{o}{.}\PYG{n}{Surface}\PYG{p}{(}\PYG{p}{)}

\PYG{g+gp}{\textgreater{}\textgreater{}\textgreater{} }\PYG{n}{main}\PYG{o}{.}\PYG{n}{informdriver}\PYG{p}{(}\PYG{p}{)}
\PYG{g+go}{0}
\end{Verbatim}

\end{fulllineitems}

\index{line() (fb.Surface static method)}

\begin{fulllineitems}
\phantomsection\label{index:fb.Surface.line}\pysiglinewithargsret{\strong{static }\bfcode{line}}{\emph{\textless{}tuple crd from\textgreater{}}, \emph{\textless{}tuple crd to\textgreater{}}}{}
or

\end{fulllineitems}

\index{poly() (fb.Surface static method)}

\begin{fulllineitems}
\phantomsection\label{index:fb.Surface.poly}\pysiglinewithargsret{\strong{static }\bfcode{poly}}{\emph{\textless{}xdata numpy array\textgreater{}}, \emph{\textless{}ydata numpy array\textgreater{}}}{}
x, y will be the points, have to be the same length and type

style = 0, 1, 2
0: solid line
1: dashed line
2: dotted line

\begin{Verbatim}[commandchars=\\\{\}]
\PYG{g+gp}{\textgreater{}\textgreater{}\textgreater{} }\PYG{k+kn}{import} \PYG{n+nn}{fbpy.fb} \PYG{k+kn}{as} \PYG{n+nn}{fb}

\PYG{g+gp}{\textgreater{}\textgreater{}\textgreater{} }\PYG{k+kn}{import} \PYG{n+nn}{numpy} \PYG{k+kn}{as} \PYG{n+nn}{np}

\PYG{g+gp}{\textgreater{}\textgreater{}\textgreater{} }\PYG{n}{x} \PYG{o}{=} \PYG{n}{np}\PYG{o}{.}\PYG{n}{arange}\PYG{p}{(}\PYG{l+m+mi}{0}\PYG{p}{,} \PYG{l+m+mi}{1}\PYG{p}{,}\PYG{l+m+mf}{0.01}\PYG{p}{)}

\PYG{g+gp}{\textgreater{}\textgreater{}\textgreater{} }\PYG{n}{y} \PYG{o}{=} \PYG{l+m+mf}{0.5}\PYG{o}{*}\PYG{n}{np}\PYG{o}{.}\PYG{n}{sin}\PYG{p}{(}\PYG{n}{x}\PYG{o}{*}\PYG{l+m+mi}{2}\PYG{o}{*}\PYG{l+m+mi}{2}\PYG{o}{*}\PYG{n}{np}\PYG{o}{.}\PYG{n}{pi}\PYG{p}{)} \PYG{o}{+} \PYG{l+m+mf}{0.5}

\PYG{g+gp}{\textgreater{}\textgreater{}\textgreater{} }\PYG{n}{main}  \PYG{o}{=} \PYG{n}{fb}\PYG{o}{.}\PYG{n}{Surface}\PYG{p}{(}\PYG{p}{)}

\PYG{g+gp}{\textgreater{}\textgreater{}\textgreater{} }\PYG{n}{subwin} \PYG{o}{=} \PYG{n}{fb}\PYG{o}{.}\PYG{n}{Surface}\PYG{p}{(}\PYG{p}{(}\PYG{l+m+mi}{0}\PYG{p}{,}\PYG{l+m+mi}{0}\PYG{p}{)}\PYG{p}{,}\PYG{p}{(}\PYG{l+m+mi}{200}\PYG{p}{,}\PYG{l+m+mi}{200}\PYG{p}{)}\PYG{p}{)}

\PYG{g+gp}{\textgreater{}\textgreater{}\textgreater{} }\PYG{n}{subwin}\PYG{o}{.}\PYG{n}{clear}\PYG{p}{(}\PYG{p}{)}
\PYG{g+go}{0}
\PYG{g+gp}{\textgreater{}\textgreater{}\textgreater{} }\PYG{n}{subwin}\PYG{o}{.}\PYG{n}{pixelstyle} \PYG{o}{=} \PYG{n}{fb}\PYG{o}{.}\PYG{n}{Pixelstyles}\PYG{o}{.}\PYG{n}{faint}

\PYG{g+gp}{\textgreater{}\textgreater{}\textgreater{} }\PYG{n}{subwin}\PYG{o}{.}\PYG{n}{poly}\PYG{p}{(}\PYG{n}{x}\PYG{p}{,} \PYG{n}{y}\PYG{p}{)}
\PYG{g+go}{0}
\PYG{g+gp}{\textgreater{}\textgreater{}\textgreater{} }\PYG{n}{subwin}\PYG{o}{.}\PYG{n}{grabsilent}\PYG{p}{(}\PYG{l+s}{"}\PYG{l+s}{./source/images/poly.png}\PYG{l+s}{"}\PYG{p}{)}
\PYG{g+go}{0}
\end{Verbatim}

\includegraphics{poly.png}

\end{fulllineitems}

\index{printxy() (fb.Surface static method)}

\begin{fulllineitems}
\phantomsection\label{index:fb.Surface.printxy}\pysiglinewithargsret{\strong{static }\bfcode{printxy}}{\emph{\textless{}tuple\textgreater{}}, \emph{\textless{}string\textgreater{}}, \emph{\textless{}size\textgreater{}}}{}
Will print text in string at position defined by tuple (x, y).

Size can be 1 or 2, where 2 prints triple sized LCD-like format

returns 0

\begin{Verbatim}[commandchars=\\\{\}]
\PYG{g+gp}{\textgreater{}\textgreater{}\textgreater{} }\PYG{k+kn}{import} \PYG{n+nn}{fbpy.fb} \PYG{k+kn}{as} \PYG{n+nn}{fb}

\PYG{g+gp}{\textgreater{}\textgreater{}\textgreater{} }\PYG{n}{main} \PYG{o}{=} \PYG{n}{fb}\PYG{o}{.}\PYG{n}{Surface}\PYG{p}{(}\PYG{p}{)}

\PYG{g+gp}{\textgreater{}\textgreater{}\textgreater{} }\PYG{n}{sub} \PYG{o}{=} \PYG{n}{fb}\PYG{o}{.}\PYG{n}{Surface}\PYG{p}{(}\PYG{p}{(}\PYG{l+m+mi}{0}\PYG{p}{,}\PYG{l+m+mi}{0}\PYG{p}{)}\PYG{p}{,}\PYG{p}{(}\PYG{l+m+mi}{800}\PYG{p}{,}\PYG{l+m+mi}{100}\PYG{p}{)}\PYG{p}{)}

\PYG{g+gp}{\textgreater{}\textgreater{}\textgreater{} }\PYG{n}{sub}\PYG{o}{.}\PYG{n}{clear}\PYG{p}{(}\PYG{p}{)}
\PYG{g+go}{0}
\PYG{g+gp}{\textgreater{}\textgreater{}\textgreater{} }\PYG{n}{sub}\PYG{o}{.}\PYG{n}{printxy}\PYG{p}{(}\PYG{p}{(}\PYG{l+m+mi}{10}\PYG{p}{,}\PYG{l+m+mi}{10}\PYG{p}{)}\PYG{p}{,}\PYG{l+s}{"}\PYG{l+s}{Hello world!}\PYG{l+s}{"}\PYG{p}{,} \PYG{l+m+mi}{2}\PYG{p}{)}
\PYG{g+go}{0 }
\PYG{g+gp}{\textgreater{}\textgreater{}\textgreater{} }\PYG{n}{sub}\PYG{o}{.}\PYG{n}{printxy}\PYG{p}{(}\PYG{p}{(}\PYG{l+m+mi}{10}\PYG{p}{,}\PYG{l+m+mi}{38}\PYG{p}{)}\PYG{p}{,}\PYG{l+s}{"}\PYG{l+s}{or a bit smaller...}\PYG{l+s}{"}\PYG{p}{,} \PYG{l+m+mi}{1}\PYG{p}{)}
\PYG{g+go}{0}
\PYG{g+gp}{\textgreater{}\textgreater{}\textgreater{} }\PYG{n}{sub}\PYG{o}{.}\PYG{n}{pixelstyle}\PYG{o}{.}\PYG{n}{color} \PYG{o}{=} \PYG{n}{fb}\PYG{o}{.}\PYG{n}{Color}\PYG{p}{(}\PYG{l+m+mi}{20}\PYG{p}{,}\PYG{l+m+mi}{20}\PYG{p}{,}\PYG{l+m+mi}{20}\PYG{p}{,}\PYG{l+m+mi}{100}\PYG{p}{)}

\PYG{g+gp}{\textgreater{}\textgreater{}\textgreater{} }\PYG{n}{sub}\PYG{o}{.}\PYG{n}{pixelstyle}\PYG{o}{.}\PYG{n}{blur} \PYG{o}{=} \PYG{l+m+mi}{2}

\PYG{g+gp}{\textgreater{}\textgreater{}\textgreater{} }\PYG{n}{sub}\PYG{o}{.}\PYG{n}{pixelstyle}\PYG{o}{.}\PYG{n}{blurradius} \PYG{o}{=} \PYG{l+m+mi}{4}

\PYG{g+gp}{\textgreater{}\textgreater{}\textgreater{} }\PYG{n}{sub}\PYG{o}{.}\PYG{n}{pixelstyle}\PYG{o}{.}\PYG{n}{sigma} \PYG{o}{=} \PYG{l+m+mi}{1}

\PYG{g+gp}{\textgreater{}\textgreater{}\textgreater{} }\PYG{n}{sub}\PYG{o}{.}\PYG{n}{printxy}\PYG{p}{(}\PYG{p}{(}\PYG{l+m+mi}{10}\PYG{p}{,}\PYG{l+m+mi}{76}\PYG{p}{)}\PYG{p}{,}\PYG{l+s}{"}\PYG{l+s}{where R them goggles...}\PYG{l+s}{"}\PYG{p}{,} \PYG{l+m+mi}{1}\PYG{p}{)}
\PYG{g+go}{0}
\PYG{g+gp}{\textgreater{}\textgreater{}\textgreater{} }\PYG{n}{sub}\PYG{o}{.}\PYG{n}{grabsilent}\PYG{p}{(}\PYG{l+s}{"}\PYG{l+s}{./source/images/printxy.png}\PYG{l+s}{"}\PYG{p}{)}
\PYG{g+go}{0               }
\end{Verbatim}

\includegraphics{printxy.png}

\end{fulllineitems}

\index{rect() (fb.Surface static method)}

\begin{fulllineitems}
\phantomsection\label{index:fb.Surface.rect}\pysiglinewithargsret{\strong{static }\bfcode{rect}}{\emph{\textless{}tuple\textgreater{}}, \emph{\textless{}tuple\textgreater{}}, \emph{\textless{}fb color\textgreater{}}, \emph{\textless{}style\textgreater{}}}{}
Will draw a rectangle @ first tuple, width and height
as in second tuple

\end{fulllineitems}

\index{set\_dotstyle() (fb.Surface method)}

\begin{fulllineitems}
\phantomsection\label{index:fb.Surface.set_dotstyle}\pysiglinewithargsret{\bfcode{set\_dotstyle}}{\emph{\textless{}dotstlyle\textgreater{}}, \emph{\textless{}blur radius\textgreater{}}}{}
dotstyle 0 : fast plot
dotstyle 1 : plot with soft alpha
dotstyle 2 : plot with blur + soft alpha

blur radius: well, 2 sigma \textasciicircum{}2 it is

\end{fulllineitems}

\index{set\_raw() (fb.Surface method)}

\begin{fulllineitems}
\phantomsection\label{index:fb.Surface.set_raw}\pysiglinewithargsret{\bfcode{set\_raw}}{\emph{sprite}}{}
puts the bitmap array into the buffer, see get\_raw.

\end{fulllineitems}

\index{snow() (fb.Surface method)}

\begin{fulllineitems}
\phantomsection\label{index:fb.Surface.snow}\pysiglinewithargsret{\bfcode{snow}}{}{}
show some noise...

\begin{Verbatim}[commandchars=\\\{\}]
\PYG{g+gp}{\textgreater{}\textgreater{}\textgreater{} }\PYG{k+kn}{import} \PYG{n+nn}{fbpy.fb} \PYG{k+kn}{as} \PYG{n+nn}{fb}

\PYG{g+gp}{\textgreater{}\textgreater{}\textgreater{} }\PYG{n}{main} \PYG{o}{=} \PYG{n}{fb}\PYG{o}{.}\PYG{n}{Surface}\PYG{p}{(}\PYG{p}{)}

\PYG{g+gp}{\textgreater{}\textgreater{}\textgreater{} }\PYG{n}{sub} \PYG{o}{=} \PYG{n}{fb}\PYG{o}{.}\PYG{n}{Surface}\PYG{p}{(}\PYG{p}{(}\PYG{l+m+mi}{0}\PYG{p}{,}\PYG{l+m+mi}{0}\PYG{p}{)}\PYG{p}{,}\PYG{p}{(}\PYG{l+m+mi}{200}\PYG{p}{,}\PYG{l+m+mi}{200}\PYG{p}{)}\PYG{p}{)}

\PYG{g+gp}{\textgreater{}\textgreater{}\textgreater{} }\PYG{n}{sub}\PYG{o}{.}\PYG{n}{clear}\PYG{p}{(}\PYG{p}{)}
\PYG{g+go}{0}
\PYG{g+gp}{\textgreater{}\textgreater{}\textgreater{} }\PYG{n}{sub}\PYG{o}{.}\PYG{n}{pixelstyle} \PYG{o}{=} \PYG{n}{fb}\PYG{o}{.}\PYG{n}{Pixelstyles}\PYG{o}{.}\PYG{n}{faint}

\PYG{g+gp}{\textgreater{}\textgreater{}\textgreater{} }\PYG{n}{sub}\PYG{o}{.}\PYG{n}{snow}\PYG{p}{(}\PYG{p}{)}
\PYG{g+go}{0}
\PYG{g+gp}{\textgreater{}\textgreater{}\textgreater{} }\PYG{n}{sub}\PYG{o}{.}\PYG{n}{grabsilent}\PYG{p}{(}\PYG{l+s}{"}\PYG{l+s}{./source/images/snow.png}\PYG{l+s}{"}\PYG{p}{)}
\PYG{g+go}{0}
\end{Verbatim}

\includegraphics{snow.png}

\end{fulllineitems}

\index{something() (fb.Surface method)}

\begin{fulllineitems}
\phantomsection\label{index:fb.Surface.something}\pysiglinewithargsret{\bfcode{something}}{}{}~
\begin{Verbatim}[commandchars=\\\{\}]
\end{Verbatim}

\begin{Verbatim}[commandchars=\\\{\}]
\PYG{g+gp}{\textgreater{}\textgreater{}\textgreater{} }\PYG{k}{print} \PYG{l+s}{"}\PYG{l+s}{Hello from a doctest..}\PYG{l+s}{"}
\PYG{g+go}{Hello from a doctest..}
\end{Verbatim}

\end{fulllineitems}

\index{update() (fb.Surface method)}

\begin{fulllineitems}
\phantomsection\label{index:fb.Surface.update}\pysiglinewithargsret{\bfcode{update}}{}{}
draws the buffered geometries. So, you need this before you actualy see 
anything

\end{fulllineitems}


\end{fulllineitems}

\index{Trafo (class in fb)}

\begin{fulllineitems}
\phantomsection\label{index:fb.Trafo}\pysigline{\strong{class }\code{fb.}\bfcode{Trafo}}
Handle two dim lintrafos for
your surface.

that is: Stretch and or Rotate

yih.

Work-flow.

You start with making an instance:

\begin{Verbatim}[commandchars=\\\{\}]
\PYG{n}{T} \PYG{o}{=} \PYG{n}{Trafo}\PYG{p}{(}\PYG{p}{)}
\end{Verbatim}

Uppon instanciation you get an unity 
transform by default.
Then decide what should happen to it.. E.g.
you want to rotate and then stretch it. 
Well, you'll define two Operators:

\begin{Verbatim}[commandchars=\\\{\}]
\PYG{n}{R} \PYG{o}{=} \PYG{n}{Trafo}\PYG{p}{(}\PYG{p}{)}
\PYG{n}{S} \PYG{o}{=} \PYG{n}{Trafo}\PYG{p}{(}\PYG{p}{)}
\PYG{n}{R}\PYG{o}{.}\PYG{n}{rotate}\PYG{p}{(}\PYG{l+m+mf}{0.1}\PYG{p}{)}           \PYG{c}{\PYGZsh{}where 0.1 is the angle in RAD}
\PYG{n}{S}\PYG{o}{.}\PYG{n}{stretch}\PYG{p}{(}\PYG{l+m+mf}{1.05}\PYG{p}{,} \PYG{l+m+mf}{1.05}\PYG{p}{)}   \PYG{c}{\PYGZsh{}ehhhr, 5\PYGZpc{} in horiz and vert}
\end{Verbatim}

Now you can iterate:

\begin{Verbatim}[commandchars=\\\{\}]
\PYG{n}{T} \PYG{o}{*}\PYG{o}{=}\PYG{n}{R}
\PYG{n}{T} \PYG{o}{*}\PYG{o}{=}\PYG{n}{S}
\end{Verbatim}

Each surface has a built in trafo fb.Surface.trafo, which is
unity or identity by default. The state of this operator is
passed to the fb driver.

Here is a full example:

\begin{Verbatim}[commandchars=\\\{\}]
\PYG{g+gp}{\textgreater{}\textgreater{}\textgreater{} }\PYG{k+kn}{import} \PYG{n+nn}{fbpy.fb} \PYG{k+kn}{as} \PYG{n+nn}{fb}

\PYG{g+gp}{\textgreater{}\textgreater{}\textgreater{} }\PYG{n}{main} \PYG{o}{=} \PYG{n}{fb}\PYG{o}{.}\PYG{n}{Surface}\PYG{p}{(}\PYG{p}{)}

\PYG{g+gp}{\textgreater{}\textgreater{}\textgreater{} }\PYG{n}{sub} \PYG{o}{=} \PYG{n}{fb}\PYG{o}{.}\PYG{n}{Surface}\PYG{p}{(}\PYG{p}{(}\PYG{l+m+mi}{100}\PYG{p}{,}\PYG{l+m+mi}{100}\PYG{p}{)}\PYG{p}{,}\PYG{p}{(}\PYG{l+m+mi}{200}\PYG{p}{,}\PYG{l+m+mi}{200}\PYG{p}{)}\PYG{p}{)}

\PYG{g+gp}{\textgreater{}\textgreater{}\textgreater{} }\PYG{n}{R} \PYG{o}{=} \PYG{n}{fb}\PYG{o}{.}\PYG{n}{Trafo}\PYG{p}{(}\PYG{p}{)}

\PYG{g+gp}{\textgreater{}\textgreater{}\textgreater{} }\PYG{n}{R}\PYG{o}{.}\PYG{n}{rotate}\PYG{p}{(}\PYG{l+m+mf}{0.1}\PYG{p}{)}

\PYG{g+gp}{\textgreater{}\textgreater{}\textgreater{} }\PYG{n}{sub}\PYG{o}{.}\PYG{n}{clear}\PYG{p}{(}\PYG{p}{)}
\PYG{g+go}{0}
\PYG{g+gp}{\textgreater{}\textgreater{}\textgreater{} }\PYG{k}{for} \PYG{n}{i} \PYG{o+ow}{in} \PYG{n+nb}{range}\PYG{p}{(}\PYG{l+m+mi}{10}\PYG{p}{)}\PYG{p}{:} 
\PYG{g+gp}{... }    \PYG{n}{sub}\PYG{o}{.}\PYG{n}{trafo}\PYG{o}{*}\PYG{o}{=}\PYG{n}{R}
\PYG{g+gp}{... }    \PYG{n}{sub}\PYG{o}{.}\PYG{n}{rect}\PYG{p}{(}\PYG{p}{(}\PYG{l+m+mi}{10}\PYG{p}{,}\PYG{l+m+mi}{10}\PYG{p}{)}\PYG{p}{,}\PYG{p}{(}\PYG{l+m+mi}{190}\PYG{p}{,}\PYG{l+m+mi}{190}\PYG{p}{)}\PYG{p}{)}
\PYG{g+go}{0}
\PYG{g+go}{0}
\PYG{g+go}{0}
\PYG{g+go}{0}
\PYG{g+go}{0}
\PYG{g+go}{0}
\PYG{g+go}{0}
\PYG{g+go}{0}
\PYG{g+go}{0}
\PYG{g+go}{0}
\PYG{g+gp}{\textgreater{}\textgreater{}\textgreater{} }\PYG{n}{sub}\PYG{o}{.}\PYG{n}{grabsilent}\PYG{p}{(}\PYG{l+s}{"}\PYG{l+s}{./source/images/rotate.png}\PYG{l+s}{"}\PYG{p}{)}
\PYG{g+go}{0}
\end{Verbatim}

\includegraphics{rotate.png}

\begin{Verbatim}[commandchars=\\\{\}]
\PYG{n}{sub}\PYG{o}{.}\PYG{n}{trafo}\PYG{o}{.}\PYG{n}{identity}\PYG{p}{(}\PYG{p}{)} \PYG{c}{\PYGZsh{}reset the transform}
\end{Verbatim}

\end{fulllineitems}

\index{Uniton (class in fb)}

\begin{fulllineitems}
\phantomsection\label{index:fb.Uniton}\pysiglinewithargsret{\strong{class }\code{fb.}\bfcode{Uniton}}{\emph{*args}, \emph{**kwargs}}{}
The Uniton is a special case of the Vulgion
and ensures inheritance of certain properties of 
the primeordial instance for all consecutive instances.

\end{fulllineitems}



\chapter{Indices and tables}
\label{index:indices-and-tables}\begin{itemize}
\item {} 
\emph{genindex}

\item {} 
\emph{modindex}

\item {} 
\emph{search}

\end{itemize}


\renewcommand{\indexname}{Python Module Index}
\begin{theindex}
\def\bigletter#1{{\Large\sffamily#1}\nopagebreak\vspace{1mm}}
\bigletter{f}
\item {\texttt{fb}}, \pageref{index:module-fb}
\end{theindex}

\renewcommand{\indexname}{Index}
\printindex
\end{document}
